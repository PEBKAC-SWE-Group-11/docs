
\documentclass[12pt, a4paper]{article}

\usepackage{graphicx}
\usepackage{amsmath}
\usepackage{pgfplots}
\pgfplotsset{compat=1.18}
\usepackage{xcolor}
\usepackage{float}
\usepackage{svg}
\usepackage[colorlinks=true, linkcolor=black, urlcolor=blue, citecolor=green]{hyperref}
\usepackage{enumitem}
\usepackage[italian]{babel}
\usepackage{lastpage}  % Pacchetto per ottenere il numero totale delle pagine
\usepackage{fancyhdr}  % Pacchetto per personalizzare l'intestazione e il piè di pagina
\usepackage{tabularx}
\usepackage[margin=1in]{geometry}
\usepackage{array}
\newcolumntype{C}[1]{>{\centering\arraybackslash}p{#1}}
\newcolumntype{L}[1]{>{\raggedright\arraybackslash}p{#1}}
\graphicspath{ {images/} {../shared/images/} }
\definecolor{unipd}{HTML}{B5121B}

\addto\captionsitalian{\renewcommand{\contentsname}{Indice}}

\setcounter{secnumdepth}{5}
\setcounter{tocdepth}{5}
\makeatletter
\newcommand\subsubsubsection{\@startsection{paragraph}{4}{\z@}{-2.5ex\@plus -1ex \@minus -.25ex}{1.25ex \@plus .25ex}{\normalfont\normalsize\bfseries}}
\newcommand\subsubsubsubsection{\@startsection{subparagraph}{5}{\z@}{-2.5ex\@plus -1ex \@minus -.25ex}{1.25ex \@plus .25ex}{\normalfont\normalsize\bfseries}}
\makeatother

\pagestyle{fancy}% Imposta lo stile di pagina su "fancy"
\fancyhf{}% Cancella intestazioni e piè di pagina
\fancyfoot[C]{\thepage{} di \pageref{LastPage}} % Imposta il piè di pagina centrale come "numero pagina di totale pagine"
\renewcommand{\headrulewidth}{0pt} % Imposta la larghezza della linea di intestazione a 0 punti

\newcommand{\data}{}
\newcommand{\titolo}{Piano di qualifica}
\newcommand{\uso}{Interno}
\newcommand{\destinatari }{
   %& Vimar S.p.A.  \\
    & Tullio Vardanega  \\
    & Riccardo Cardin  }
\newcommand{\abstractcontent}{ ... }

\begin{document}

\begin{minipage}[]{0.3\textwidth}
\includesvg[width=\linewidth]{pebkac.svg} 
\end{minipage}
\hspace{0.1\textwidth}
\begin{minipage}[]{0.6\textwidth}
  {\Large \textbf{PEBKAC}} \\
  Email: \href{mailto:pebkacswe@gmail.com}{pebkacswe@gmail.com} \\
  Gruppo: 11
\end{minipage}

\bigskip

\begin{minipage}[]{0.3\textwidth}
\includesvg[width=\linewidth]{logo_unipd.svg} 
\end{minipage}
\hspace{0.1\textwidth}
\begin{minipage}[]{0.6\textwidth}
  \textcolor{unipd}{
    \textbf{Università degli Studi di Padova} \\
    Corso di Laurea: Informatica \\
    Corso: Ingegneria del Software \\
    Anno Accademico: 2024/2025
  }
\end{minipage}


\bigskip
\bigskip
\bigskip
\begin{center}
  \Huge\textbf{Preventivo dei Costi e Assunzione degli Impegni}

  \Large\textbf{\data}
\end{center}

\bigskip


\begin{center}
\textbf{Informazioni sul documento}: \\
\vspace{0.5cm}

\begin{tabular}{r|l}
% VERBALE
\textbf{Responsabile} &  Matteo Gerardin\\ 
\textbf{Verificatore} &  Davide Martinelli\\ 
\textbf{Redattore} &     Matteo Piron\\ 
    \textbf{Uso} & Interno \\ 
    \textbf{Destinatari} \destinatari \\
\end{tabular}

\vfill

\textbf{Abstract}: \\
\vspace{0.5cm}
Esito della riunione in cui si è discusso l'utilizzo di determinati modelli di LLM e discussione sul cambio dei ruoli
\end{center}


\bigskip
\newpage

\section*{Registro delle modifiche}
\begin{table}[H]
    \begin{tabular}{|c|c|c|c|c|}
        \hline
         \textbf{Versione} &  \textbf{Data} &  \textbf{Autore} &  \textbf{Ruolo} & \textbf{Descrizione} \\
          \hline
          &  &  & Responsabile & Approvazione e rilascio\\
          \hline
          &  &  &  &  \\
          \hline
          0.0.2& 25/10/24 & Derek Gusatto & Amministratore & Aggiunta domanda §5 e link  \\
          \hline
          0.0.1& 24/10/24 & Derek Gusatto & Amministratore & Stesura \\
          \hline
    \end{tabular}
\end{table}
\newpage
\tableofcontents
\newpage
\section{Introduzione}
\subsection{Scopo del documento}
Il presente documento ha l’obiettivo di definire le \textit{best practices}\textsubscript{G}  e il \textit{way of working}\textsubscript{G} che i componenti del team \textit{PEBKAC} hanno l’obbligo di rispettare per l’intero svolgimento del progetto. L'intento è di garantire un metodo di lavoro omogeneo, verificabile
e migliorabile nel tempo. La creazione delle norme è progressiva e incrementale nel tempo per consentire al team di apportare aggiornamenti continui in risposta alle esigenze e alle problematiche incorse durante lo sviluppo dell'intero progetto.

\subsection{Scopo del prodotto}
Il progetto ``Vimar GENIALE" mira a sviluppare un'applicazione intelligente che supporti installatori elettrici nell'uso di dispositivi Vimar\textsubscript{G}, facilitando l'accesso alle informazioni tecniche sui prodotti, rispondendo a domande poste in linguaggio naturale.
La tecnologia alla base prevede l'uso di modelli di \textit{LLM}\textsubscript{G} e di tecniche \textit{RAG}\textsubscript{G}, con una struttura di gestione basata su \textit{container}\textsubscript{G} e integrata in un ambiente cloud.
Il sistema include tre componenti principali: una \textit{applicativo web responsive}\textsubscript{G}, un \textit{applicativo server}\textsubscript{G} e un'\textit{infrastruttura cloud-ready}\textsubscript{G}. 
\subsection{Glossario}
Per evitare ambiguità relative al linguaggio utilizzato nei documenti, viene fornito il Glossario V1.0.0, nel quale si possono trovare tutte le definizioni di termini che hanno un significato specifico che vuole essere disambiguato. Tali termini sono marcati con una G a pedice. 
\subsection{Riferimenti}
\subsubsection{Riferimenti normativi}
\begin{itemize}
    \item Regolamento del progetto didattico\\ \href{https://www.math.unipd.it/~tullio/IS-1/2024/Dispense/PD1.pdf}{https://www.math.unipd.it/~tullio/IS-1/2024/Dispense/PD1.pdf} \\ (Ultimo accesso 2024-11-14)
    \item ISO/IEC 12207:1995 Information technology - Software life cycle processes \\ \href{https://www.math.unipd.it/~tullio/IS-1/2010/Approfondimenti/A03.pdf}{https://www.math.unipd.it/~tullio/IS-1/2010/Approfondimenti/A03.pdf}\\ (Ultimo accesso 2024-11-14)

\end{itemize}

\subsubsection{Riferimenti informativi}
\begin{itemize}
    \item Capitolato C2 \\ \href{https://www.math.unipd.it/~tullio/IS-1/2024/Dispense/PD1.pdf}{https://www.math.unipd.it/~tullio/IS-1/2024/Dispense/PD1.pdf}\\ (Ultimo accesso 2024-11-14)
    \item Capitolato C2 - slides \\ \href{https://www.math.unipd.it/~tullio/IS-1/2024/Dispense/PD1.pdf}{https://www.math.unipd.it/~tullio/IS-1/2024/Dispense/PD1.pdf}\\ (Ultimo accesso 2024-11-14)
    \item Documentazione\textsubscript{G} GitHub\textsubscript{G} \\ \href{https://docs.github.com/en}{https://docs.github.com/en}\\ (Ultimo accesso 2024-11-14)
    
\end{itemize}
\newpage
\section{Qualità di processo}
La qualità di processo si basa sull’idea che, per realizzare un prodotto conforme a specifici standard qualitativi, è essenziale monitorare e migliorare regolarmente i processi che lo generano. Questo principio si applica all’intera gamma di attività, pratiche e metodologie impiegate durante il ciclo di vita del software.
In altre parole, la qualità dei processi ha l'obiettivo di andare a conformare la qualità del prodotto in modo tale da garantire sempre che gli standard definidi nel documento \textit{Norme di Progetto} vengano rispettati ed eventualmente migliorati. Di seguito sono elencate le metriche che il team si impegna a rispettare per garantire l’eccellenza nei processi. La sigla \textit{MPC} sta ad indicare le metriche di processo.
\subsection{Processi primari}
\subsubsection{Fornitura}

\begin{table}[H]
    \centering
    \begin{tabularx}{\textwidth}{>{\hsize=0.7\hsize}X|X|>{\centering\arraybackslash}X|>{\hsize=0.8\hsize}>{\centering\arraybackslash}X}
        \textbf{Metrica} & \textbf{Descrizione} & \textbf{Valore Accettabile} & \textbf{Valore Ideale} \\ \hline
        
         MPC - CV& Cost variance& \(\ge -7.5\%\) & \(\ge 0\%\) \\ \hline
         MPC - PV& Planned Value& \(\ge 0\) & \(\le BAC\) \\ \hline
         MPC - EV& Earned Value& \(\ge 0\) & \(\le EAC\) \\ \hline
         MPC - AC& Actual Cost& \(\ge 0\)  & \(\le EAC\) \\ \hline
         MPC - CPI& Cost Performance Index & tra 0.95 e 1.05& 1 \\ \hline
         MPC - EAC& Estimated At Completion & deviazione del 5\% del BAC & BAC \\ \hline 
         MPC - VAC& Variance At Completion & deviazione del 10\% del BAC & 0\% \\ \hline
         MPC - ETC& Estimated To Completion & \(\ge 0\) & \(\le EAC\) \\ \hline
         MPC - SV& Schedule Variance & deviazione del 10\% del BAC & 0\% \\ \hline
         MPC - BV& Budget Variance & deviazione del 10\% del BAC & 0\% \\ 
         
    \end{tabularx}
    \caption{Documenti del ciclo di vita del prodotto SW}
\end{table}

\subsubsection{Sviluppo}
\subsubsubsection{Codifica}

\begin{table}[H]
    \centering
    \begin{tabularx}{\textwidth}{>{\hsize=0.7\hsize}X|X|>{\centering\arraybackslash}X|>{\hsize=0.8\hsize}>{\centering\arraybackslash}X}
   
        \textbf{Metrica} & \textbf{Descrizione} & \textbf{Valore accettabile} & \textbf{Valore ideale}  \\
        \hline
        MPC-SC & Statement Coverage &  0 & \(\ge 100\%\) \\
        
    \end{tabularx}
    \caption{Tabella riguardante le metriche per il processo di codifica}
\end{table}

\subsection{Processi di supporto}
\subsubsubsection{Documentazione}

\begin{table}[H]
    \centering
    \begin{tabularx}{\textwidth}{>{\hsize=0.7\hsize}X|X|>{\centering\arraybackslash}X|>{\hsize=0.8\hsize}>{\centering\arraybackslash}X}
   
        \textbf{Metrica} & \textbf{Descrizione} & \textbf{Valore accettabile} & \textbf{Valore ideale}  \\
        \hline
        MPC-IG & Indice Gulpease & \(\ge65\%\) & 100 \\
        \hline
        MPC-CO & Correttezza Ortografica & 0 & 0 \\

        
    \end{tabularx}
    \caption{Tabella riguardante le metriche per il processo di documentazione}
\end{table}

\subsubsubsection{Gestione della qualità}

\begin{table}[H]
    \centering
    \begin{tabularx}{\textwidth}{>{\hsize=0.7\hsize}X|X|>{\centering\arraybackslash}X|>{\hsize=0.8\hsize}>{\centering\arraybackslash}X}
   
        \textbf{Metrica} & \textbf{Descrizione} & \textbf{Valore accettabile} & \textbf{Valore ideale}  \\
        \hline
        MPC-MNS & Metriche Non Soddisfatte & \(\le3\) & 0\\
        
    \end{tabularx}
    \caption{Tabella riguardante le metriche per il processo di gestione delle qualità}
\end{table}


\newpage
\section{Qualità del prodotto}
La qualità del prodotto si concentra sulla valutazione del software sviluppato, ponendo l'accento su caratteristiche come facilità d'uso, funzionalità, affidabilità, capacità di manutenzione e, in senso più ampio, sulle prestazioni complessive del prodotto. L'obiettivo principale del team è anche quello non solo di soddisfare le attese del cliente fornendo un prodotto software che implementi le funzionalità volute, ma che lo faccia seguendo precisi standard di qualità. Vengono quindi fornite di seguito le metriche che il team si impegna a soddisfare nel contesto della qualità del prodotto. La sigla MPD sta per metriche di prodotto come definito nel documento \textit{Norme di progetto v 1.0.0}
\subsection{Funzionalità}

\begin{table}[H]
    \centering
    \begin{tabularx}{\textwidth}{>{\hsize=0.7\hsize}X|X|>{\centering\arraybackslash}X|>{\hsize=0.8\hsize}>{\centering\arraybackslash}X}
   
        \textbf{Metrica} & \textbf{Descrizione} & \textbf{Valore accettabile} & \textbf{Valore ideale}  \\
        \hline
        MPC-ROS & Requisiti Obbligatori Soddisfatti & \(100\%\) & 0\\
        \hline
        MPC-RDS & Requisiti Desiderabili Soddisfatti & \(\ge0\%\) & \(\ge75\%\)\\
        \hline
        MPC-RPS & Requisiti Opzionali Soddisfatti & \(\ge0\%\) & \(\ge50\%\)\\
        
    \end{tabularx}
    \caption{Tabella riguardante le metriche per la funzionalità del prodotto}
\end{table}

\subsection{Manutenibilità}

\begin{table}[H]
    \centering
    \begin{tabularx}{\textwidth}{>{\hsize=0.7\hsize}X|X|>{\centering\arraybackslash}X|>{\hsize=0.8\hsize}>{\centering\arraybackslash}X}
   
        \textbf{Metrica} & \textbf{Descrizione} & \textbf{Valore accettabile} & \textbf{Valore ideale}  \\
        \hline
        MPC-SFIN & Structure Fan In & da determinare & da determinare\\
        \hline
        MPC-SFOUT & Structure Fan Out & da determinare & da determinare\\
        
    \end{tabularx}
    \caption{Tabella riguardante le metriche per la manutenibilità del prodotto}
\end{table}

\subsection{Affidabilità}

\begin{table}[H]
    \centering
    \begin{tabularx}{\textwidth}{>{\hsize=0.7\hsize}X|X|>{\centering\arraybackslash}X|>{\hsize=0.8\hsize}>{\centering\arraybackslash}X}
   
        \textbf{Metrica} & \textbf{Descrizione} & \textbf{Valore accettabile} & \textbf{Valore ideale}  \\
        \hline
        MPC-PTCP & Passed Test Cases Percentage & \(\ge80\%\) & \(100\%\)\\
        \hline
        MPC-CC & Code Coverage & \(\ge80\%\) & \(100\%\)\\
        
    \end{tabularx}
    \caption{Tabella riguardante le metriche per la affidabilità del prodotto}
\end{table}

\subsection{Efficienza}

\begin{table}[H]
    \centering
    \begin{tabularx}{\textwidth}{>{\hsize=0.7\hsize}X|X|>{\centering\arraybackslash}X|>{\hsize=0.8\hsize}>{\centering\arraybackslash}X}
   
        \textbf{Metrica} & \textbf{Descrizione} & \textbf{Valore accettabile} & \textbf{Valore ideale}  \\
        \hline
        MPC-TDE & Tempo Di Elaborazione & da determinare & da determinare\\
        
    \end{tabularx}
    \caption{Tabella riguardante le metriche per l'efficienza del prodotto}
\end{table}
\newpage
\input{contents/strategie_di_testing}
\newpage
\section{Cruscotto delle metriche}
\subsection{Qualità di processo - Fornitura}
\subsection{Qualità di processo - Documentazione}
\subsection{Qualità di processo - Gestione delle Qualità}
\subsection{Considerazioni finali in vista della revisione RTB}

\end{document}
