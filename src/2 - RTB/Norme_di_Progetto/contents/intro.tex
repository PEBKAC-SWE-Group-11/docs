\section{Introduzione}
\subsection{Scopo del documento}
Il presente documento ha l’obiettivo di definire le \textit{best practices}\textsubscript{G}  e il \textit{way of working}\textsubscript{G} che i componenti del team \textit{PEBKAC} hanno l’obbligo di rispettare per l’intero svolgimento del progetto. L'intento è di garantire un metodo di lavoro omogeneo, verificabile
e migliorabile nel tempo. La creazione delle norme è progressiva e incrementale nel tempo per consentire al team di apportare aggiornamenti continui in risposta alle esigenze e alle problematiche incorse durante lo sviluppo dell'intero progetto.

\subsection{Scopo del prodotto}
Il progetto ``Vimar GENIALE" mira a sviluppare un'applicazione intelligente che supporti installatori elettrici nell'uso di dispositivi Vimar\textsubscript{G}, facilitando l'accesso alle informazioni tecniche sui prodotti, rispondendo a domande poste in linguaggio naturale.
La tecnologia alla base prevede l'uso di modelli di \textit{LLM}\textsubscript{G} e di tecniche \textit{RAG}\textsubscript{G}, con una struttura di gestione basata su \textit{container}\textsubscript{G} e integrata in un ambiente cloud.
Il sistema include tre componenti principali: una \textit{applicativo web responsive}\textsubscript{G}, un \textit{applicativo server}\textsubscript{G} e un'\textit{infrastruttura cloud-ready}\textsubscript{G}. 
\subsection{Glossario}
Per evitare ambiguità relative al linguaggio utilizzato nei documenti, viene fornito il Glossario V1.0.0, nel quale si possono trovare tutte le definizioni di termini che hanno un significato specifico che vuole essere disambiguato. Tali termini sono marcati con una G a pedice. 
\subsection{Riferimenti}
\subsubsection{Riferimenti normativi}
\begin{itemize}
    \item Regolamento del progetto didattico\\ \href{https://www.math.unipd.it/~tullio/IS-1/2024/Dispense/PD1.pdf}{https://www.math.unipd.it/~tullio/IS-1/2024/Dispense/PD1.pdf} \\ (Ultimo accesso 2024-11-14)
    \item ISO/IEC 12207:1995 Information technology - Software life cycle processes \\ \href{https://www.math.unipd.it/~tullio/IS-1/2010/Approfondimenti/A03.pdf}{https://www.math.unipd.it/~tullio/IS-1/2010/Approfondimenti/A03.pdf}\\ (Ultimo accesso 2024-11-14)

\end{itemize}

\subsubsection{Riferimenti informativi}
\begin{itemize}
    \item Capitolato C2 \\ \href{https://www.math.unipd.it/~tullio/IS-1/2024/Dispense/PD1.pdf}{https://www.math.unipd.it/~tullio/IS-1/2024/Dispense/PD1.pdf}\\ (Ultimo accesso 2024-11-14)
    \item Capitolato C2 - slides \\ \href{https://www.math.unipd.it/~tullio/IS-1/2024/Dispense/PD1.pdf}{https://www.math.unipd.it/~tullio/IS-1/2024/Dispense/PD1.pdf}\\ (Ultimo accesso 2024-11-14)
    \item Documentazione\textsubscript{G} GitHub\textsubscript{G} \\ \href{https://docs.github.com/en}{https://docs.github.com/en}\\ (Ultimo accesso 2024-11-14)
    
\end{itemize}