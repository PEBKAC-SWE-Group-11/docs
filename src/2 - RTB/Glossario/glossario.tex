
\documentclass[12pt, a4paper]{article}

\usepackage{graphicx}
\usepackage{xcolor}
\usepackage{float}
\usepackage{svg}
\usepackage[colorlinks=true, linkcolor=black, urlcolor=blue, citecolor=green]{hyperref}
\usepackage{enumitem}
\usepackage[italian]{babel}
\usepackage{lastpage}  % Pacchetto per ottenere il numero totale delle pagine
\usepackage{fancyhdr}  % Pacchetto per personalizzare l'intestazione e il piè di pagina
\usepackage[margin=1in]{geometry}
\usepackage{array}
\newcolumntype{C}[1]{>{\centering\arraybackslash}p{#1}}
\newcolumntype{L}[1]{>{\raggedright\arraybackslash}p{#1}}
\graphicspath{ {images/} {../shared/images/} }
\definecolor{unipd}{HTML}{B5121B}

\addto\captionsitalian{\renewcommand{\contentsname}{Indice}}


\pagestyle{fancy}% Imposta lo stile di pagina su "fancy"
\fancyhf{}% Cancella intestazioni e piè di pagina
\fancyfoot[C]{\thepage{} di \pageref{LastPage}} % Imposta il piè di pagina centrale come "numero pagina di totale pagine"
\renewcommand{\headrulewidth}{0pt} % Imposta la larghezza della linea di intestazione a 0 punti

%\newcommand{\data}{GG mese AAAA}
\newcommand{\titolo}{Glossario}
%\newcommand{\responsabile}{Responsabile}
\newcommand{\verificatore}{
    & Nome1 Cognome1 \\
    & Nome2 Cognome2 \\
    & Nome3 Cognome3
}
\newcommand{\redattore}{
    & Nome1 Cognome1 \\
    & Nome2 Cognome2 \\
    & Nome3 Cognome3
}
\newcommand{\uso}{Esterno}
\newcommand{\destinatari }{
    & Tullio Vardanega  \\
    & Riccardo Cardin \\  
    & Vimar S.p.A.
}
\newcommand{\abstractcontent}{abstract ...}

\begin{document}
\begin{minipage}[]{0.3\textwidth}
\includesvg[width=\linewidth]{pebkac.svg} 
\end{minipage}
\hspace{0.1\textwidth}
\begin{minipage}[]{0.6\textwidth}
  {\Large \textbf{PEBKAC}} \\
  Email: \href{mailto:pebkacswe@gmail.com}{pebkacswe@gmail.com} \\
  Gruppo: 11
\end{minipage}

\bigskip

\begin{minipage}[]{0.3\textwidth}
\includesvg[width=\linewidth]{logo_unipd.svg} 
\end{minipage}
\hspace{0.1\textwidth}
\begin{minipage}[]{0.6\textwidth}
  \textcolor{unipd}{
    \textbf{Università degli Studi di Padova} \\
    Corso di Laurea: Informatica \\
    Corso: Ingegneria del Software \\
    Anno Accademico: 2024/2025
  }
\end{minipage}


\bigskip
\bigskip
\bigskip
\begin{center}
  \Huge\textbf{Preventivo dei Costi e Assunzione degli Impegni}

  \Large\textbf{\data}
\end{center}

\bigskip


\begin{center}
\textbf{Informazioni sul documento}: \\
\vspace{0.5cm}

\begin{tabular}{r|l}
% VERBALE
\textbf{Responsabile} &  Matteo Gerardin\\ 
\textbf{Verificatore} &  Davide Martinelli\\ 
\textbf{Redattore} &     Matteo Piron\\ 
    \textbf{Uso} & Interno \\ 
    \textbf{Destinatari} \destinatari \\
\end{tabular}

\vfill

\textbf{Abstract}: \\
\vspace{0.5cm}
Esito della riunione in cui si è discusso l'utilizzo di determinati modelli di LLM e discussione sul cambio dei ruoli
\end{center}


\bigskip
\newpage

\section*{Registro delle modifiche}
\begin{table}[H]
    \begin{tabular}{|c|c|c|c|c|}
        \hline
         \textbf{Versione} &  \textbf{Data} &  \textbf{Autore} &  \textbf{Ruolo} & \textbf{Descrizione} \\
          \hline
          &  &  & Responsabile & Approvazione e rilascio\\
          \hline
          &  &  &  &  \\
          \hline
          0.0.2& 25/10/24 & Derek Gusatto & Amministratore & Aggiunta domanda §5 e link  \\
          \hline
          0.0.1& 24/10/24 & Derek Gusatto & Amministratore & Stesura \\
          \hline
    \end{tabular}
\end{table}
\newpage
\tableofcontents
\newpage
\section{Introduzione}
\subsection{Scopo del Documento}
Questo documento si propone di descrivere la pianificazione e il coordinamento delle attività indispensabili per l'esecuzione del progetto. Vengono trattati in dettaglio elementi fondamentali come l'analisi dei rischi, il modello di sviluppo scelto, la programmazione delle attività, l'assegnazione dei ruoli, oltre a una stima dei costi e delle risorse richieste.
\subsection{Scopo del Prodotto} 
L'obiettivo del prodotto è consentire agli installatori che utilizzano le soluzioni dell'azienda \textit{proponente$_G$}, Vimar S.p.A., di accedere rapidamente a informazioni testuali e grafiche relative ai prodotti presenti sul sito ufficiale.
\subsection{Glossario}
Per garantire chiarezza e prevenire fraintendimenti legati alla terminologia utilizzata nel documento, si è scelto di includere un \texttt{Glossario} che raccolga le definizioni dei termini.\\ I termini presenti nel glossario saranno scritti in \textit{corsivo} e marcati con una $_G$ a pedice.
\subsection{Riferimenti}
\subsubsection{Riferimenti  normativi}  
\begin{itemize}
    \item \texttt{Norme di Progetto vX.X.X}\\
    \item \textbf{PD1 - Regolamento del progetto didattico} \\
    \url{https://www.math.unipd.it/~tullio/IS-1/2024/Dispense/PD1.pdf} 
    \item \textbf{Capitolato d'Appalto C2}: Vimar GENIALE \\
    \url{https://www.math.unipd.it/~tullio/IS-1/2024/Progetto/C2.pdf}
\end{itemize}
\subsubsection{Riferimenti informativi}
\begin{itemize}
    \item \textbf{T2-I Processi di Ciclo di Vita del Software} \\
    \url{https://www.math.unipd.it/~tullio/IS-1/2024/Dispense/T02.pdf}
    \item \textbf{T4-Gestione di Progetto} \\
    \url{https://www.math.unipd.it/~tullio/IS-1/2024/Dispense/T04.pdf}
    \item \texttt{Glossario vX.X.X} \\
\end{itemize}
\subsection{Preventivo Iniziale}
Il preventivo iniziale, presentato durante la fase di candidatura, è disponibile al seguente link: \href{https://pebkac-swe-group-11.github.io/assets/pdf/Preventivo_Costi_Assunzione_Impegni_V2.0.0.pdf}{Preventivo Iniziale}.\\Nel documento si specifica che il costo stimato per il progetto ammonta a \textbf{12850€} e che il gruppo prevede di completare il prodotto entro la data \textbf{14 marzo 2025}.
\newpage
\section{A}
\begin{itemize}
    \item 
\end{itemize}
\newpage
\section{B}
\begin{itemize}
    \item 
\end{itemize}
\newpage
\input{contents/alfabet/c}
\newpage
\input{contents/alfabet/d}
\newpage
\input{contents/alfabet/e}
\newpage
\section{F}
\begin{itemize}
    \item 
\end{itemize}
\newpage
\input{contents/alfabet/g}
\newpage
\input{contents/alfabet/h}
\newpage
\section{I}
\begin{itemize}
    \item 
\end{itemize}
\newpage
\section{J}
\begin{itemize}
    \item 
\end{itemize}
\newpage
\section{K}
\begin{itemize}
    \item 
\end{itemize}
\newpage
\section{L}
\begin{itemize}
    \item 
\end{itemize}
\newpage
\section{M}
\begin{itemize}
    \item 
\end{itemize}
\newpage
\section{N}
\begin{itemize}
    \item 
\end{itemize}
\newpage
\section{O}
\begin{itemize}
    \item 
\end{itemize}
\newpage
\section{P}
\begin{itemize}
    \item 
\end{itemize}
\newpage
\input{contents/alfabet/q}
\newpage
\section{R}
\begin{itemize}
    \item 
\end{itemize}
\newpage
\input{contents/alfabet/s}
\newpage
\section{T}
\begin{itemize}
    \item 
\end{itemize}
\newpage
\section{U}
\begin{itemize}
    \item 
\end{itemize}
\newpage
\section{V}
\begin{itemize}
    \item 
\end{itemize}
\newpage
\section{W}
\begin{itemize}
    \item 
\end{itemize}
\newpage
\section{X}
\begin{itemize}
    \item 
\end{itemize}
\newpage
\input{contents/alfabet/y}
\newpage
\section{Z}
\begin{itemize}
    \item 
\end{itemize}
\newpage


\end{document}
