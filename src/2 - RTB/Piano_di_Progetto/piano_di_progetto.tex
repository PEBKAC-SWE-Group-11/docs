
\documentclass[12pt, a4paper]{article}

\usepackage{graphicx}
\usepackage{xcolor}
\usepackage{float}
\usepackage{svg}
\usepackage[colorlinks=true, linkcolor=black, urlcolor=blue, citecolor=green]{hyperref}
\usepackage{enumitem}
\usepackage[italian]{babel}
\usepackage{lastpage}  % Pacchetto per ottenere il numero totale delle pagine
\usepackage{fancyhdr}  % Pacchetto per personalizzare l'intestazione e il piè di pagina
\usepackage[margin=1in]{geometry}
\usepackage{array}
\newcolumntype{C}[1]{>{\centering\arraybackslash}p{#1}}
\newcolumntype{L}[1]{>{\raggedright\arraybackslash}p{#1}}
\usepackage{tikz}
\usepackage[table,xcdraw]{xcolor}
\newcommand\colorrow{\rowcolor{lightgray}\cellcolor{white}}
\graphicspath{ {images/} {../shared/images/} }
\definecolor{unipd}{HTML}{B5121B}

\addto\captionsitalian{\renewcommand{\contentsname}{Indice}}


\pagestyle{fancy}% Imposta lo stile di pagina su "fancy"
\fancyhf{}% Cancella intestazioni e piè di pagina
\fancyfoot[C]{\thepage{} di \pageref{LastPage}} % Imposta il piè di pagina centrale come "numero pagina di totale pagine"
\renewcommand{\headrulewidth}{0pt} % Imposta la larghezza della linea di intestazione a 0 punti

\newcommand{\data}{}
\newcommand{\titolo}{Piano di Progetto}
%\newcommand{\responsabile}{Responsabile}
\newcommand{\verificatore}{
    & Nome1 Cognome1 \\
    & Nome2 Cognome2 \\
    & Nome3 Cognome3
}
\newcommand{\redattore}{
    & Nome1 Cognome1 \\
    & Nome2 Cognome2 \\
    & Nome3 Cognome3
}
\newcommand{\uso}{& Esterno}
\newcommand{\destinatari}{
    & Tullio Vardanega \\
    & Riccardo Cardin \\
    & Vimar S.p.A.
}
\newcommand{\abstractcontent}{abstract ...}

\begin{document}

\begin{minipage}[]{0.3\textwidth}
\includesvg[width=\linewidth]{pebkac.svg} 
\end{minipage}
\hspace{0.1\textwidth}
\begin{minipage}[]{0.6\textwidth}
  {\Large \textbf{PEBKAC}} \\
  Email: \href{mailto:pebkacswe@gmail.com}{pebkacswe@gmail.com} \\
  Gruppo: 11
\end{minipage}

\bigskip

\begin{minipage}[]{0.3\textwidth}
\includesvg[width=\linewidth]{logo_unipd.svg} 
\end{minipage}
\hspace{0.1\textwidth}
\begin{minipage}[]{0.6\textwidth}
  \textcolor{unipd}{
    \textbf{Università degli Studi di Padova} \\
    Corso di Laurea: Informatica \\
    Corso: Ingegneria del Software \\
    Anno Accademico: 2024/2025
  }
\end{minipage}


\bigskip
\bigskip
\bigskip
\begin{center}
  \Huge\textbf{Preventivo dei Costi e Assunzione degli Impegni}

  \Large\textbf{\data}
\end{center}

\bigskip


\begin{center}
\textbf{Informazioni sul documento}: \\
\vspace{0.5cm}

\begin{tabular}{r|l}
% VERBALE
\textbf{Responsabile} &  Matteo Gerardin\\ 
\textbf{Verificatore} &  Davide Martinelli\\ 
\textbf{Redattore} &     Matteo Piron\\ 
    \textbf{Uso} & Interno \\ 
    \textbf{Destinatari} \destinatari \\
\end{tabular}

\vfill

\textbf{Abstract}: \\
\vspace{0.5cm}
Esito della riunione in cui si è discusso l'utilizzo di determinati modelli di LLM e discussione sul cambio dei ruoli
\end{center}


\bigskip
\newpage

\section*{Registro delle modifiche}
\begin{table}[H]
    \begin{tabular}{|c|c|c|c|c|}
        \hline
         \textbf{Versione} &  \textbf{Data} &  \textbf{Autore} &  \textbf{Ruolo} & \textbf{Descrizione} \\
          \hline
          &  &  & Responsabile & Approvazione e rilascio\\
          \hline
          &  &  &  &  \\
          \hline
          0.0.2& 25/10/24 & Derek Gusatto & Amministratore & Aggiunta domanda §5 e link  \\
          \hline
          0.0.1& 24/10/24 & Derek Gusatto & Amministratore & Stesura \\
          \hline
    \end{tabular}
\end{table}
\newpage
\tableofcontents
\newpage
\section{Introduzione}
\subsection{Scopo del Documento}
Questo documento si propone di descrivere la pianificazione e il coordinamento delle attività indispensabili per l'esecuzione del progetto. Vengono trattati in dettaglio elementi fondamentali come l'analisi dei rischi, il modello di sviluppo scelto, la programmazione delle attività, l'assegnazione dei ruoli, oltre a una stima dei costi e delle risorse richieste.
\subsection{Scopo del Prodotto} 
L'obiettivo del prodotto è consentire agli installatori che utilizzano le soluzioni dell'azienda \textit{proponente$_G$}, Vimar S.p.A., di accedere rapidamente a informazioni testuali e grafiche relative ai prodotti presenti sul sito ufficiale.
\subsection{Glossario}
Per garantire chiarezza e prevenire fraintendimenti legati alla terminologia utilizzata nel documento, si è scelto di includere un \texttt{Glossario} che raccolga le definizioni dei termini.\\ I termini presenti nel glossario saranno scritti in \textit{corsivo} e marcati con una $_G$ a pedice.
\subsection{Riferimenti}
\subsubsection{Riferimenti  normativi}  
\begin{itemize}
    \item \texttt{Norme di Progetto vX.X.X}\\
    \item \textbf{PD1 - Regolamento del progetto didattico} \\
    \url{https://www.math.unipd.it/~tullio/IS-1/2024/Dispense/PD1.pdf} 
    \item \textbf{Capitolato d'Appalto C2}: Vimar GENIALE \\
    \url{https://www.math.unipd.it/~tullio/IS-1/2024/Progetto/C2.pdf}
\end{itemize}
\subsubsection{Riferimenti informativi}
\begin{itemize}
    \item \textbf{T2-I Processi di Ciclo di Vita del Software} \\
    \url{https://www.math.unipd.it/~tullio/IS-1/2024/Dispense/T02.pdf}
    \item \textbf{T4-Gestione di Progetto} \\
    \url{https://www.math.unipd.it/~tullio/IS-1/2024/Dispense/T04.pdf}
    \item \texttt{Glossario vX.X.X} \\
\end{itemize}
\subsection{Preventivo Iniziale}
Il preventivo iniziale, presentato durante la fase di candidatura, è disponibile al seguente link: \href{https://pebkac-swe-group-11.github.io/assets/pdf/Preventivo_Costi_Assunzione_Impegni_V2.0.0.pdf}{Preventivo Iniziale}.\\Nel documento si specifica che il costo stimato per il progetto ammonta a \textbf{12850€} e che il gruppo prevede di completare il prodotto entro la data \textbf{14 marzo 2025}.
\newpage
\section{Analisi dei Rischi}

\subsection{Rischi Tecnologici}

\begin{tabular}{|L{4cm}|L{9.7cm}|}
    \hline
    \textbf{Id. Rischio} & \textbf{RT1} \\
    \hline
    \textbf{Rischio} & Inesperienza \\
    \hline
    \textbf{Descrizione} & La mancanza di esperienza con le tecnologie richieste dal progetto può comportare ritardi nello sviluppo, errori nel codice e difficoltà nell'utilizzo degli strumenti. \\
    \hline
    \textbf{Pericolosità} & Alta \\
    \hline
    \textbf{Occorrenza} & Alta \\
    \hline
    \textbf{Piano di intervento} & I membri in ``inesperti" avranno tempo di studiare le nuove tecnologie ed eventualmente saranno affiancati a membri più esperti. \\
    \hline
\end{tabular}
\\[30pt]
\begin{tabular}{|L{4cm}|L{9.7cm}|}
    \hline
    \textbf{Id. Rischio} & \textbf{RT2} \\
    \hline
    \textbf{Rischio} & Problemi con software di terze parti \\
    \hline
    \textbf{Descrizione} & L'utilizzo di software o librerie esterne può comportare malfunzionamenti, incompatibilità e difficoltà di integrazione. \\
    \hline
    \textbf{Pericolosità} & Alta \\
    \hline
    \textbf{Occorrenza} & Bassa \\
    \hline
    \textbf{Piano di intervento} & L'adozione di un sistema di versionamento dei file, ``backup" regolari e la replica dei dati. \\
    \hline
\end{tabular}
\\[30pt]
\begin{tabular}{|L{4cm}|L{9.7cm}|}
    \hline
    \textbf{Id. Rischio} & \textbf{RT3} \\
    \hline
    \textbf{Rischio} & Basse prestazioni hardware \\
    \hline
    \textbf{Descrizione} & Le risorse hardware limitate dei PC personali potrebbero risultare insufficienti per condurre test approfonditi. \\
    \hline
    \textbf{Pericolosità} & Media \\
    \hline
    \textbf{Occorrenza} & Media \\
    \hline
    \textbf{Piano di intervento} & Ottimizzazione del codice, semplificazione delle funzionalità del progetto, adozione di strategie di test meno onerose. \\
    \hline
\end{tabular}

\subsection{Rischi Organizzativi}

\begin{tabular}{|L{4cm}|L{9.7cm}|}
    \hline
    \textbf{Id. Rischio} & \textbf{RO1} \\
    \hline
    \textbf{Rischio} & Imprecisioni nella pianificazione \\
    \hline
    \textbf{Descrizione} & La sottostima o sovrastima dei tempi e delle risorse per completare le attività può portare a ritardi nello sviluppo, sforamento del budget e stress. \\
    \hline
    \textbf{Pericolosità} & Alta \\
    \hline
    \textbf{Occorrenza} & Alta \\
    \hline
    \textbf{Piano di intervento} & Pianificazione flessibile, revisioni periodiche del Piano di Progetto, confronto con il proponente, riassegnazione di compiti e suddivisione delle attività in task più piccole. \\
    \hline
\end{tabular}
\\[30pt]
\begin{tabular}{|L{4cm}|L{9.7cm}|}
    \hline
    \textbf{Id. Rischio} & \textbf{RO2} \\
    \hline
    \textbf{Rischio} & Impegni personali (e accademici) \\
    \hline
    \textbf{Descrizione} & La difficoltà nel conciliare gli impegni del progetto con quelli personali, in particolare durante gli esami, può comportare ritardi e assenze. \\
    \hline
    \textbf{Pericolosità} & Alta \\
    \hline
    \textbf{Occorrenza} & Alta \\
    \hline
    \textbf{Piano di intervento} & Comunicazione tempestiva, pianificazione di periodi di "stallo", ridistribuzione dei compiti con successivo recupero dell'eventuale ``debito lavorativo". \\
    \hline
\end{tabular}
\\[30pt]
\begin{tabular}{|L{4cm}|L{9.7cm}|}
    \hline
    \textbf{Id. Rischio} & \textbf{RO3} \\
    \hline
    \textbf{Rischio} & Variazione dei requisiti di progetto \\
    \hline
    \textbf{Descrizione} & Durante l'implementazione del progetto potrebbero emergere cambiamenti nei requisiti, i quali potrebbero determinare una deviazione delle attività pianificate. \\
    \hline
    \textbf{Pericolosità} & Media \\
    \hline
    \textbf{Occorrenza} & Alta \\
    \hline
    \textbf{Piano di intervento} & Preparare un'analisi dettagliata dei requisiti all'inizio del progetto e attuare tempestivamente le eventuali misure correttive necessarie. \\
    \hline
\end{tabular}
\\[30pt]
\begin{tabular}{|L{4cm}|L{9.7cm}|}
    \hline
    \textbf{Id. Rischio} & \textbf{RO4} \\
    \hline
    \textbf{Rischio} & Mal interpretazione dei requisiti \\
    \hline
    \textbf{Descrizione} & Un'interpretazione errata dei requisiti del progetto può portare alla realizzazione di un prodotto non conforme alle aspettative. \\
    \hline
    \textbf{Pericolosità} & Media \\
    \hline
    \textbf{Occorrenza} & Bassa \\
    \hline
    \textbf{Piano di intervento} & Documentazione chiara e dettagliata, comunicazione costante con il cliente. \\
    \hline
\end{tabular}

\subsection{Rischi Interni al Gruppo}

\begin{tabular}{|L{4cm}|L{9.7cm}|}
    \hline
    \textbf{Id. Rischio} & \textbf{RG1} \\
    \hline
    \textbf{Rischio} & Problemi di comunicazione interna \\
    \hline
    \textbf{Descrizione} & La difficoltà di comunicazione efficace all'interno del team può portare a incomprensioni, errori e ritardi. \\
    \hline
    \textbf{Pericolosità} & Alta \\
    \hline
    \textbf{Occorrenza} & Alta \\
    \hline
    \textbf{Piano di intervento} & Strumenti di comunicazione adeguati, meeting regolari, clima collaborativo e definizione chiara dei ruoli. \\
    \hline
\end{tabular}
\\[30pt]
\begin{tabular}{|L{4cm}|L{9.7cm}|}
    \hline
    \textbf{Id. Rischio} & \textbf{RG2} \\
    \hline
    \textbf{Rischio} & Rischio di conflitti interni \\
    \hline
    \textbf{Descrizione} & Data la durata del progetto, conflitti tra i membri del team possono sorgere. \\
    \hline
    \textbf{Pericolosità} & Bassa \\
    \hline
    \textbf{Occorrenza} & Bassa \\
    \hline
    \textbf{Piano di intervento} & Clima di rispetto reciproco, un mediatore per le controversie. \\
    \hline
\end{tabular}
\newpage
\section{Modello di Sviluppo}
Dopo un'approfondita analisi e un confronto, il gruppo ha deciso di adottare il modello Agile per la gestione del progetto.\\
A differenza di metodologie più tradizionali, questo approccio ampiamente utilizzato nello sviluppo software, si basa in un ciclo continuo di pianificazione, implementazione e verifica.\\
Nello specifico, il team ha stabilito una struttura operativa organizzata in cicli bisettimanali. La scelta di Agile è stata motivata dai seguenti benefici principali: 
\begin{itemize}[topsep=1pt, itemsep=1pt]
    \item \textbf{Gestione efficace dei rischi}: la breve durata dei cicli consente di individuare eventuali criticità in tempi rapidi, riducendo sia l’impatto delle problematiche sia il rischio complessivo di fallimento;
    \item \textbf{Partecipazione del team}: la struttura di questo modello favorisce il coinvolgimento attivo del team, grazie alla frequente trasformazione delle attività svolte in risultati concreti;
    \item \textbf{Massima ``trasparenza"}: il modello consente di presentare regolarmente i progressi agli stakeholder, facilitando il controllo sull'avanzamento dei lavori;
    \item \textbf{Adattabilità ai cambiamenti}: grazie alla struttura iterativa, Agile permette di gestire agevolmente modifiche ai requisiti del progetto, rispondendo tempestivamente a nuovi scenari o esigenze impreviste.
\end{itemize}
\newpage
\input{contents/Periodi}
\end{document}