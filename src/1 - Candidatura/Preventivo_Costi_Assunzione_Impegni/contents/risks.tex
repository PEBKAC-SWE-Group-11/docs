\section{Rischi previsti e mitigazione}
\subsection{Periodi di rallentamento}
Il gruppo ha stabilito che possono esserci dei rallentamenti nel:
\begin{itemize}
    \item \textbf{Periodo Natalizio} (dal 23/12/2024 al 6/1/2025) per impegni personali;
    \item \textbf{Sessione Invernale} (dal 20/1/2025 al 22/2/2025) a causa degli esami che alcuni membri devono sostenere.
\end{itemize}
\subsection{Mitigazione dei rischi}
In questa sezione sono evidenziati i rischi che il gruppo ha previsto come possibili e le operazioni che il gruppo si impegna a mettere in atto per la loro mitigazione.

\subsubsection{Sottostima dei tempi}
Nel caso in cui il gruppo si accorga di una sottostima del tempo necessario per il completamento di un requisito questo va prima di tutto segnalato a tutto il gruppo in tempi brevi, sfruttando i corretti canali di comunicazione. Anche in questo caso saranno i membri del gruppo con del tempo disponibile ad assumere la responsabilità di aiutare e assistere al fine di accorciare il tempo di completamento del requisito.
\\
Nel caso in cui si abbia una sottostima, il gruppo, prima di procedere con i successivi task, si impegna nel riallineamento impiegando tutte le risorse disponibili per assicurare il soddisfacimento dei requisiti, escludendo ovviamente le risorse impegnate in attività ad alta priorità o che non possono essere interrotte.


\subsubsection{Difficoltà nella realizazione}
Durante lo sviluppo del progetto il gruppo potrebbe incontrare delle difficoltà nella realizzazione di qualche attività: i componenti interessati dovranno rendere note le problematiche al resto del gruppo e, in caso di disponibilità, i colleghi che hanno tempo disponibile saranno incaricati di fornire assistenza.
Successivamente, che l’attività venga portata a termine o meno, le difficoltà emerse dovranno essere riportate nella riunione successiva più imminente, così da procedere con una sua suddivisione in sotto-task più semplici, con l'obiettivo di ottimizzare l’efficienza operativa e ridurre i tempi di completamento complessivi.

\subsubsection{Conoscenze non omogenee}
Come ovvio, all'interno del team le competenze non sono omogenee e questo potrebbe causare una discrepanza, in determinate attvità, tra la stima oraria calcolata per il completamento e l'effettivo numero di ore necessarie per la realizzazione.
\\
Per mitigare tale rischio il gruppo sfrutterà la collaborazione interna, grazie alla quale chi possiede una certa competenza in determinati ambiti si mette a disposizione per coloro le cui abilità in tale ambito sono ancora limitate. 

\subsubsection{Impegni personali}
Il prospetto orario stilato tiene in considerazione uno scarto per eventuali imprevisti.
Nel caso in cui un membro del gruppo non riesca ad eseguire i task a lui assegnati entro la scadenza, a causa di impegni personali esterni, il gruppo coprirà l'intervallo di tempo non produttivo con una suddivisione omogenea tra i restanti membri delle attività rimaste in sospeso, fermo restando che il componente inadempiente si assume la responsabilità di recuperare le ore dedicate dal team per sopperire alla sua assenza totale o parziale.



