\subsection{Capitolato C8 - Requirement Tracker - Plug-in VS Code}
     \subsubsection{Informazioni generali}
        \begin{itemize}
            \item \textbf{Titolo}: Requirement Tracker - Plug-in VS Code
            \item \textbf{Proponente}: Bluewind s.r.l.
            \item \textbf{Committente}: Prof. Vardanega T., Prof. Cardin R.
        \end{itemize}
    \subsubsection{Obiettivo}
    Lo scopo del progetto è quello di creare un plug-in per VS code per migliorare il processo di sviluppo software, in particolare in ambienti embedded.
Il plug-in mira a eliminare la necessità di verifiche manuali nel tracciamento e nella verifica dei requisiti rispetto al codice implementato, l’IA sarà quindi implementata per analizzare la scrittura dei requisiti e suggerire miglioramenti, assicurando che siano chiari e specifici per gli sviluppatori. 

     \subsubsection{Dominio Applicativo}
Le funzioni che dovrà possedere questo sistema sono:
\begin{itemize}
    \item Il sistema mira a essere estendibile a più tecnologie e a integrare nuove funzionalità;
    \item  Il plug-in dovrà soddisfare alcune norme come  ISO 26262 e IEC 61508;
    \item Il plug-in dovrà analizzare il codice sorgente per trovare corrispondenze con i requisiti derivati da specifiche fornite al progetto e in caso fornire correzioni per migliorare chiarezza, completezza e specificità;
    \item  I risultati dell’analisi verranno visualizzati all’interno dell’interfaccia di Visual Studio Code, con annessi strumenti per migliorare la navigazione.
\end{itemize}
    \subsubsection{Tecnologie}
    Il progetto utilizza:

\begin{itemize}
    \item Visual Studio Code Extension Api per costruire un'architettura modulare, che consenta l'aggiunta di
nuove funzionalità in maniera semplice;
    \item Python o Node.js per l'integrazione con le API AI;
    \item API REST per connettersi a modelli di AI per l'analisi del codice e dei requisiti;
    \item Modelli AI pre-addestrati per analisi semantiche.
\end{itemize}
    \subsubsection{Punti di forza}
   \begin{itemize}
    \item Il progetto è pensato per essere esteso a nuovi linguaggi di programmazione;
    \item L’automatizzazione del tracciamento dei requisiti nel codice cerca di eliminare la necessità di verifiche manuali, riducendo il rischio di errori;
    \item L’analisi dell’IA offre suggerimenti(anche se non sempre utili o pertinenti) per migliorare la chiarezza e la precisione, aiutando a evitare ambiguità che possono far perdere tempo e risorse al team di sviluppo.
\end{itemize}
    \subsubsection{Punti deboli}
   \begin{itemize}
    \item L’integrazione del processo AI per suggerire correzioni e miglioramenti può non sempre andare ad aiutare lo sviluppatore, in quanto l’AI non sempre può essere a conoscenza di tutto il background del progetto preso in considerazione;
    \item Nel progetto viene menzionata l’integrazione con modelli AI come GPT o altri, non è pero chiaro come l’AI verrà gestita, in quanto viene indicata la possibilità di fornire un feedback sull’implementazione del codice e sulla scarsa qualità dei requisiti, ma non è stato definito come l’AI andra a lavorare su questa parte di codice, fornendo limiti o linee guida;
    \item Il supporto è limitato ai linguaggi C/C++;
    \item L’idea di dare dei feedback automatici mediante l’AI può generare alcuni problemi, spesso questi feedback richiedono tecnicismi e molto spesso specifici, un analisi automatica potrebbe non essere sufficiente e anzi sviare il programmatore;
    \item Non viene specificato come valutare le norme di sicurezza  ISO 26262 o IEC 61508;
    \item Non viene citata la GUI richiesta per la visualizzazione dei risultati;
    \item  Viene richiesta una copertura dei test superiore all’80\% ma non vengono menzionati quali test saranno implementati.
\end{itemize}
    \subsubsection{Conclusioni}
   Questo capitolato non è stato preso in considerazione durante la fase di selezione dei capitolati dato che l’ambito a cui appartiene non ha interessato particolarmente i membri del gruppo, facendo sì che l’attenzione ricadesse su altri capitolati.
