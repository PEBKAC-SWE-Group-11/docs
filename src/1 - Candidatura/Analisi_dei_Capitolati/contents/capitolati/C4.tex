\subsection{Capitolato C4 - NearYounSmart custom advertising platform}
     \subsubsection{Informazioni generali}
        \begin{itemize}
            \item \textbf{Titolo}: NearYounSmart custom advertising platform
            \item \textbf{Proponente}: SyncLab
            \item \textbf{Committente}: Prof. Vardanega T., Prof. Cardin R.
        \end{itemize}
     \subsubsection{Obiettivo}
    Il progetto verte sulla realizzazione di un sistema pubblicitario mirato all’utente che grazie a dati raccolti da esso (come posizione, abitudini e interessi) riesce a fornire degli annunci personalizzati. L’utilizzo di un sistema del genere vorrebbe essere impiegato su un veicolo tramite un piccolo dispositivo simulabile da una web app. Gli annunci forniti non hanno solo l'obiettivo di rispecchiare al meglio gli interessi dell’utente ma anche quello di fornire una migliore visibilità e targeting agli inserzionisti.
     \subsubsection{Dominio Applicativo}
    Il proponente chiede la realizzazione di una web-app che simuli un piccolo schermo e dove il suo funzionamento si basi sull’utilizzo di data stream processing grazie al raccoglimento dei dati da sensori (come GPS) e alla loro elaborazione tramite LLM. I risultati vengono poi visualizzati in una mappa con a fianco pop-up con gli annunci.
    \subsubsection{Tecnologie}
    Il progetto utilizza:
    \begin{itemize}
        \item Python per l'utilizzo di framework per la simulazione dei dati
        \item Utilizzo di broker per disaccoppiare lo stream di informazioni provenienti dai simulatori.
        \item Apache Airflow,Apache NiFi per lo stream processing.
        \item LangChain, FLow per l'utilizzo di uno strumento in grado di processare i messaggi in input e fornire una risposta tramite LLM.
        \item PostGIS,ClickHouse,Timescale per l'utilizzo di database o altre tipologie di storage in grado di soddisfare le esigenze specifiche di progetto. 
        \item Superset,Grafana,Tableau per l'utilizzo di uno o più strumenti per la data visualization delle informazioni.
    \end{itemize}

    \subsubsection{Punti di forza}

    \begin{itemize}
        \item La piattaforma sfrutta LLM per generare messaggi pubblicitari personalizzati, adattandoli a dati comportamentali, di localizzazione e preferenze degli utenti in tempo reale, il che aumenta significativamente la rilevanza e l’efficacia degli annunci.
        \item  Utilizzando dati GPS e informazioni di contesto, la piattaforma consente di mostrare annunci in base alla posizione esatta e al momento più adatto, migliorando le probabilità di interazione e conversione.
        \item L’uso di strumenti come Apache Kafka e Apache Spark consente una gestione robusta e scalabile dei dati in tempo reale, garantendo continuità e adattabilità della piattaforma a grandi volumi di dati, migliorando così la performance.
    \end{itemize}
    \subsubsection{Punti deboli}
    
    \begin{itemize}
        \item La gestione dei flussi dati in tempo reale e l’elaborazione tramite LLM richiedono elevate risorse computazionali, il che può aumentare i costi operativi, soprattutto se si considerano utenti su larga scala.
        \item La misurazione del successo degli annunci pubblicitari può risultare complessa; ottenere dati accurati sulle conversioni può richiedere ulteriori integrazioni o sistemi di tracciamento e potrebbe variare in modo significativo tra gli utenti.
    \end{itemize}
    
    \subsubsection{Conclusioni}

    Il progetto presenta grandi vantaggi in termini di rilevanza e capacità di ottimizzare l'esperienza pubblicitaria, tutta via non ha riscontrato un enorme successo tra i componenti del gruppo.
